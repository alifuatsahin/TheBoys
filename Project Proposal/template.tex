
\documentclass[12pt]{article}

\usepackage{graphicx}

\usepackage[a4paper, top=2cm, bottom=2.5cm, left=3cm,right=3cm,headsep=0.8cm,footskip=1.2cm,]{geometry}

\setlength{\parindent}{15pt}
\setlength{\parskip}{1em}

\usepackage[utf8]{inputenc}
\usepackage[T1]{fontenc}

\usepackage{XCharter}

\usepackage{fancyhdr} % Required for customizing headers and footers

\fancypagestyle{firstpage}{
	\fancyhf{}
	\renewcommand{\headrulewidth}{0pt}
	\renewcommand{\footrulewidth}{1pt}
}

\fancypagestyle{subsequentpages}{%
	\fancyhf{}
	\renewcommand{\headrulewidth}{1pt} 
	\renewcommand{\footrulewidth}{1pt}
}

\AtBeginDocument{\thispagestyle{firstpage}}
\pagestyle{subsequentpages}

%----------------------------------------------------------------------------------------

\begin{document}

\thispagestyle{empty}
\vspace*{-1.5cm} \medskip 
\begin{minipage}{0.295\textwidth} 
\raggedright
\normalsize
TheBoys \hfill\\ 
\end{minipage}
\begin{minipage}{0.35\textwidth} 
\centering 
\large 
Autonomous Ground Vehicle Project\\ 
\normalsize  
\end{minipage}
\begin{minipage}{0.295\textwidth} 
\raggedleft
{\normalsize{\today}}\\ % Date
\end{minipage}
\medskip
\bigskip

\rule{\linewidth}{1pt}

\begin{center}
\Large
\textbf{Problem Definition}
\end{center}


\smallskip 

Transporting goods and materials from one place to another is a crucial component of many industries. However, current methods of transportation, such as trucks and trains, are often limited by their size, weight, and inflexibility in navigating different terrain. In addition, transportation costs can be high, with energy consumption and maintenance adding to the overall expenses. To address these issues, the development of an autonomous combinable ground vehicle that carries loads and is adjustable to different ground conditions can provide a more efficient and cost-effective solution. 

Overall, this project aims to create a more efficient and cost-effective solution for transporting goods and materials. By developing an autonomous combinable ground vehicle that is adjustable to different ground conditions, the project has the potential to revolutionize the transportation industry and address some of the limitations of current transportation methods.

\subsection*{Project Requirements:}

\begin{enumerate}
\item \textbf{Compactness:} The robot should be compact in size and easy to maneuver in tight spaces.
\item \textbf{Low cost:} The robot should have a low production and maintenance cost
\item \textbf{All-Terrain Capability:} The robot should be able to ride on various terrains such as rough surfaces, uneven terrains, and inclined surfaces.
\item \textbf{ROS Compatibility:} The robot should be ROS (Robot Operating System) compatible, allowing for easy integration into existing robotics frameworks.
\item \textbf{Easy to Manufacture:} The robot should be easy to manufacture using common materials and manufacturing techniques.
\item \textbf{Open to Improvement:} The design should be open to improvement, allowing for future upgrades and modifications to enhance the robot's capabilities.
\item \textbf{Safety:} The robot should comply with safety regulations and standards.
\end{enumerate}


\subsection*{Project Specifications:}

\begin{enumerate}
\item The robot footprint shouldn’t be larger than 50x50 cm.
\item Cost of the robot should not exceed \$1500.
\item The robot should be able to
\item The vehicle should be ROS compatible and have the ability to be adaptable to improvements with ROS middleware.
\item All the mechanical components of the robot should be manufactured within a day by 3-D printers and basic machining methods.
\item The vehicle should have a flexible software architecture to allow for easy modification and customization.
\item The robot should have collision avoidance, emergency stop mechanisms and onboard power supply in case of power failure to ensure safe operation.

\end{enumerate}


%----------------------------------------------------------------------------------------

\end{document}
